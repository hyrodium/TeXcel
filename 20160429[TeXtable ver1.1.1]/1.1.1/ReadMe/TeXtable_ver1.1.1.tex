\documentclass{jsarticle}
\usepackage{amsmath, amssymb}
\usepackage{bm}
\usepackage[dvipdfmx]{graphicx}
\usepackage{wrapfig}
\usepackage{longtable}
\usepackage{hhline}
\usepackage{multirow}
\usepackage{caption}
\usepackage[dvipdfmx,colorlinks=true,linkcolor=blue,filecolor=blue,urlcolor=blue]{hyperref}
\usepackage{atbegshi}
\ifnum 42146=\euc"A4A2
  \AtBeginShipoutFirst{\special{pdf:tounicode EUC-UCS2}}
\else
  \AtBeginShipoutFirst{\special{pdf:tounicode 90ms-RKSJ-UCS2}}
\fi
%\setlength{\oddsidemargin}{-0.54cm}
%\setlength{\textwidth}{52zw}
\setlength{\textheight}{720pt}
%\setlength{\columnsep}{1cm}
\setlength{\voffset}{-2.2cm}
\newcommand\pair[1]{\left(#1\right)}
\newcommand\transpose[1]{\,{\vphantom{#1}}^t\!#1}
\allowdisplaybreaks[4]

\title{TeXtable ver 1.1.1}
\author{Yuto Horikawa}
\begin{document}
\maketitle
\vspace{-3zw}
\section{概要}
TeXtableはExcelの表から\TeX の\verb|table|や\verb|matrix|のコードを生成するためのVBAマクロである.
\url{http://holyhedron.tumblr.com/TeXtable}からダウンロードできる.
\section{使い方}
基本的な使い方は同梱の\verb|TeXtable_ver1.1.1.xls|のHow to useに記載した通りであるからそちらを参照されたい.
質問$\cdot$要望などあれば\href{https://twitter.com/Hyrodium}{@Hyrodium (twitter)}または\href{http://hyrodium.tumblr.com/}{hyrodium (tumblr)}まで.
\section{サンプル}
\verb|TeXtable_ver1.1.1.xls|付属のSample dataの実行結果を最後に記載しておこう.

Sample 1は論理和・論理積の表である.
Excelのセル内に\$ \$で囲んで数式を打ち込めば表中に数式が反映される.
\vspace{-0.5em}
\begin{table}[htb]
	\captionsetup{labelformat=empty,labelsep=none}
	\caption{Sample 1}
	\label{sample1}
	\centering
	\begin{tabular}{|c|c|c|c|}\hline
		$A$	&$B$	&$A\land B$	&$A\lor B$	\\ \hline
		0	&0	&0	&0	\\ \hline
		0	&1	&0	&1	\\ \hline
		1	&0	&0	&1	\\ \hline
		1	&1	&1	&1	\\ \hline
	\end{tabular}
\end{table}
%\newpage

Sample 2は数独である.
%筆者はこのためだけに数独の問題を作成するためのプログラムをCで作成した.
TeXtableには枠線の引き方や文字寄せの設定機能はない.
このような細かい点に関しては生成されたコードをユーザーが修正することを想定している.
この例では\verb|hhline|パッケージを使用して2重線を引いている.
\vspace{-0.5em}
\begin{table}[htb]
	\captionsetup{labelformat=empty,labelsep=none}
	\caption{Sample 2}
	\label{sample2}
	\centering
	\begin{tabular}{|c|c|c||c|c|c||c|c|c|}\hline
		5	&	&	&	&3	&7	&	&	&9	\\ \hline
		8	&	&	&	&	&1	&	&4	&	\\ \hline
		1	&	&2	&6	&	&	&	&	&5	\\ \hhline{|=|=|=#=|=|=#=|=|=|}
			&7	&	&	&1	&	&5	&	&	\\ \hline
			&	&	&	&8	&	&	&	&	\\ \hline
		2	&	&	&3	&	&	&	&	&	\\ \hhline{|=|=|=#=|=|=#=|=|=|}
			&2	&4	&	&	&	&	&	&6	\\ \hline
			&1	&	&	&7	&	&	&8	&	\\ \hline
			&	&	&	&	&6	&1	&	&	\\ \hline
	\end{tabular}
\end{table}

\newpage
Sample 3のような複雑な表(表?)も簡単に作成できる.
結合されたセルを扱うには\verb|multirow|パッケージを読み込む必要がある.
TeXtableで作成される表のセルは全て中央寄せになっている.
\vspace{-0.5em}
\begin{table}[htb]
	\captionsetup{labelformat=empty,labelsep=none}
	\caption{Sample 3}
	\label{sample3}
	\centering
	\begin{tabular}{|c|c|c|c|}\hline
		\multicolumn{1}{|c|}{\multirow{3}{*}{a}}	&b	&\multicolumn{1}{|c|}{\multirow{2}{*}{c}}	&d	\\ \cline{2-2}\cline{4-4}
		\multicolumn{1}{|c|}{}	&e	&\multicolumn{1}{|c|}{}	&f	\\ \cline{2-4}
		\multicolumn{1}{|c|}{}	&\multicolumn{3}{|c|}{\multirow{2}{*}{g}}	\\ \cline{1-1}
		\multicolumn{1}{|c|}{\multirow{4}{*}{h}}	&\multicolumn{3}{|c|}{}	\\ \cline{2-4}
		\multicolumn{1}{|c|}{}	&i	&j	&\multicolumn{1}{|c|}{\multirow{2}{*}{k}}	\\ \cline{2-3}
		\multicolumn{1}{|c|}{}	&l	&\multicolumn{1}{|c|}{\multirow{2}{*}{m}}	&\multicolumn{1}{|c|}{}	\\ \cline{2-2}\cline{4-4}
		\multicolumn{1}{|c|}{}	&n	&\multicolumn{1}{|c|}{}	&o	\\ \hline
		p	&q	&\multicolumn{1}{|c|}{\multirow{2}{*}{r}}	&\multicolumn{1}{|c|}{\multirow{5}{*}{s}}	\\ \cline{1-2}
		\multicolumn{2}{|c|}{\multirow{2}{*}{t}}	&\multicolumn{1}{|c|}{}	&\multicolumn{1}{|c|}{}	\\ \cline{3-3}
		\multicolumn{2}{|c|}{}	&u	&\multicolumn{1}{|c|}{}	\\ \cline{1-3}
		v	&\multicolumn{2}{|c|}{\multirow{1}{*}{w}}	&\multicolumn{1}{|c|}{}	\\ \cline{1-3}
		x	&y	&z	&\multicolumn{1}{|c|}{}	\\ \hline
	\end{tabular}
\end{table}

Sample 4は行列(matrix)である.
行列でも使い方は表の場合と殆ど同様である.
結合されたセルにも対応している.
セルに \$ と入力すれば適切な点々$\cdots,\  \vdots\ , \ddots$に変換される.
\begin{align}
	\tag{Sample 4}
	\begin{pmatrix}
		a_{11}	&\cdots	&a_{1\ j-1}	&a_{1\ j+1}	&\cdots	&a_{1n}	\\
		\vdots	&\ddots	&\vdots	&\vdots	&\ddots	&\vdots	\\
		a_{i-1\ 1}	&\cdots	&a_{i-1\ j-1}	&a_{i-1\ j+1}	&\cdots	&a_{i-1\ n}	\\
		a_{i+1\ 1}	&\cdots	&a_{i+1\ j-1}	&a_{i+1\ j+1}	&\cdots	&a_{i+1\ n}	\\
		\vdots	&\ddots	&\vdots	&\vdots	&\ddots	&\vdots	\\
		a_{m1}	&\cdots	&a_{m\ j-1}	&a_{m\ j+1}	&\cdots	&a_{mn}
	\end{pmatrix}, \quad
	\begin{pmatrix}
		a_{11}	&\cdots	&a_{1m}	&\multicolumn{3}{c}{\multirow{3}{*}{$\hspace{-0.4em}\raisebox{-1.5em}{\text{\huge0}}$}}	\\
		\vdots	&\ddots	&\vdots	&\multicolumn{3}{c}{}	\\
		a_{m1}	&\cdots	&a_{mm}	&\multicolumn{3}{c}{}	\\
		\multicolumn{3}{c}{\multirow{3}{*}{$\hspace{-0.4em}\raisebox{-1.5em}{\text{\huge0}}$}}	&b_{11}	&\cdots	&b_{1n}	\\
		\multicolumn{3}{c}{}	&\vdots	&\ddots	&\vdots	\\
		\multicolumn{3}{c}{}	&b_{n1}	&\cdots	&b_{nn}
	\end{pmatrix}
\end{align}
%\newpage
\\

Sample 5は三角関数表である.
この例のように, 表がページを跨ぐ場合にはlongtableを使用するのが有効である.
longtableを使用する際には\verb|longtable|パッケージを読み込むことを忘れてはならない.
数値を含む場合はExcel上でROUND関数, TEXT関数を使って四捨五入の後にテキスト形式に変換しておくと良い.
このようにして小数点以下最後の桁の$0$が表示されない問題等が解消される.

\begin{longtable}{|c|c|c|c|}
	\captionsetup{labelformat=empty,labelsep=none}
	\caption{Sample 5}
	\label{sample5}
	\\ \hline\centering
	$\theta(^\circ)$	&$\sin(\theta)$	&$\cos(\theta)$	&$\tan(\theta)$	\\ \hline\endhead
	0	&0.00000000	&1.00000000	&0.00000000	\\ \hline
	1	&0.01745241	&0.99984770	&0.01745506	\\ \hline
	2	&0.03489950	&0.99939083	&0.03492077	\\ \hline
	3	&0.05233596	&0.99862953	&0.05240778	\\ \hline
	4	&0.06975647	&0.99756405	&0.06992681	\\ \hline
	5	&0.08715574	&0.99619470	&0.08748866	\\ \hline
	6	&0.10452846	&0.99452190	&0.10510424	\\ \hline
	7	&0.12186934	&0.99254615	&0.12278456	\\ \hline
	8	&0.13917310	&0.99026807	&0.14054083	\\ \hline
	9	&0.15643447	&0.98768834	&0.15838444	\\ \hline
	10	&0.17364818	&0.98480775	&0.17632698	\\ \hline
	11	&0.19080900	&0.98162718	&0.19438031	\\ \hline
	12	&0.20791169	&0.97814760	&0.21255656	\\ \hline
	13	&0.22495105	&0.97437006	&0.23086819	\\ \hline
	14	&0.24192190	&0.97029573	&0.24932800	\\ \hline
	15	&0.25881905	&0.96592583	&0.26794919	\\ \hline
	16	&0.27563736	&0.96126170	&0.28674539	\\ \hline
	17	&0.29237170	&0.95630476	&0.30573068	\\ \hline
	18	&0.30901699	&0.95105652	&0.32491970	\\ \hline
	19	&0.32556815	&0.94551858	&0.34432761	\\ \hline
	20	&0.34202014	&0.93969262	&0.36397023	\\ \hline
	21	&0.35836795	&0.93358043	&0.38386404	\\ \hline
	22	&0.37460659	&0.92718385	&0.40402623	\\ \hline
	23	&0.39073113	&0.92050485	&0.42447482	\\ \hline
	24	&0.40673664	&0.91354546	&0.44522869	\\ \hline
	25	&0.42261826	&0.90630779	&0.46630766	\\ \hline
	26	&0.43837115	&0.89879405	&0.48773259	\\ \hline
	27	&0.45399050	&0.89100652	&0.50952545	\\ \hline
	28	&0.46947156	&0.88294759	&0.53170943	\\ \hline
	29	&0.48480962	&0.87461971	&0.55430905	\\ \hline
	30	&0.50000000	&0.86602540	&0.57735027	\\ \hline
	31	&0.51503807	&0.85716730	&0.60086062	\\ \hline
	32	&0.52991926	&0.84804810	&0.62486935	\\ \hline
	33	&0.54463904	&0.83867057	&0.64940759	\\ \hline
	34	&0.55919290	&0.82903757	&0.67450852	\\ \hline
	35	&0.57357644	&0.81915204	&0.70020754	\\ \hline
	36	&0.58778525	&0.80901699	&0.72654253	\\ \hline
	37	&0.60181502	&0.79863551	&0.75355405	\\ \hline
	38	&0.61566148	&0.78801075	&0.78128563	\\ \hline
	39	&0.62932039	&0.77714596	&0.80978403	\\ \hline
	40	&0.64278761	&0.76604444	&0.83909963	\\ \hline
	41	&0.65605903	&0.75470958	&0.86928674	\\ \hline
	42	&0.66913061	&0.74314483	&0.90040404	\\ \hline
	43	&0.68199836	&0.73135370	&0.93251509	\\ \hline
	44	&0.69465837	&0.71933980	&0.96568877	\\ \hline
	45	&0.70710678	&0.70710678	&1.00000000	\\ \hline
	46	&0.71933980	&0.69465837	&1.03553031	\\ \hline
	47	&0.73135370	&0.68199836	&1.07236871	\\ \hline
	48	&0.74314483	&0.66913061	&1.11061251	\\ \hline
	49	&0.75470958	&0.65605903	&1.15036841	\\ \hline
	50	&0.76604444	&0.64278761	&1.19175359	\\ \hline
	51	&0.77714596	&0.62932039	&1.23489716	\\ \hline
	52	&0.78801075	&0.61566148	&1.27994163	\\ \hline
	53	&0.79863551	&0.60181502	&1.32704482	\\ \hline
	54	&0.80901699	&0.58778525	&1.37638192	\\ \hline
	55	&0.81915204	&0.57357644	&1.42814801	\\ \hline
	56	&0.82903757	&0.55919290	&1.48256097	\\ \hline
	57	&0.83867057	&0.54463904	&1.53986496	\\ \hline
	58	&0.84804810	&0.52991926	&1.60033453	\\ \hline
	59	&0.85716730	&0.51503807	&1.66427948	\\ \hline
	60	&0.86602540	&0.50000000	&1.73205081	\\ \hline
	61	&0.87461971	&0.48480962	&1.80404776	\\ \hline
	62	&0.88294759	&0.46947156	&1.88072647	\\ \hline
	63	&0.89100652	&0.45399050	&1.96261051	\\ \hline
	64	&0.89879405	&0.43837115	&2.05030384	\\ \hline
	65	&0.90630779	&0.42261826	&2.14450692	\\ \hline
	66	&0.91354546	&0.40673664	&2.24603677	\\ \hline
	67	&0.92050485	&0.39073113	&2.35585237	\\ \hline
	68	&0.92718385	&0.37460659	&2.47508685	\\ \hline
	69	&0.93358043	&0.35836795	&2.60508906	\\ \hline
	70	&0.93969262	&0.34202014	&2.74747742	\\ \hline
	71	&0.94551858	&0.32556815	&2.90421088	\\ \hline
	72	&0.95105652	&0.30901699	&3.07768354	\\ \hline
	73	&0.95630476	&0.29237170	&3.27085262	\\ \hline
	74	&0.96126170	&0.27563736	&3.48741444	\\ \hline
	75	&0.96592583	&0.25881905	&3.73205081	\\ \hline
	76	&0.97029573	&0.24192190	&4.01078093	\\ \hline
	77	&0.97437006	&0.22495105	&4.33147587	\\ \hline
	78	&0.97814760	&0.20791169	&4.70463011	\\ \hline
	79	&0.98162718	&0.19080900	&5.14455402	\\ \hline
	80	&0.98480775	&0.17364818	&5.67128182	\\ \hline
	81	&0.98768834	&0.15643447	&6.31375151	\\ \hline
	82	&0.99026807	&0.13917310	&7.11536972	\\ \hline
	83	&0.99254615	&0.12186934	&8.14434643	\\ \hline
	84	&0.99452190	&0.10452846	&9.51436445	\\ \hline
	85	&0.99619470	&0.08715574	&11.43005230	\\ \hline
	86	&0.99756405	&0.06975647	&14.30066626	\\ \hline
	87	&0.99862953	&0.05233596	&19.08113669	\\ \hline
	88	&0.99939083	&0.03489950	&28.63625328	\\ \hline
	89	&0.99984770	&0.01745241	&57.28996163	\\ \hline
	90	&1.00000000	&0.00000000	&$\mathrm{-}$	\\ \hline
\end{longtable}
\newpage
 
\vspace{25zw}
\begin{table}[htb]
	\captionsetup{labelformat=empty,labelsep=none}
	\caption{数独のこたえ}
	\centering
	\begin{tabular}{|r|r|r||r|r|r||r|r|r|}\hline
		5	&4	&6	&8	&3	&7	&2	&1	&9	\\ \hline
		8	&9	&7	&5	&2	&1	&6	&4	&3	\\ \hline
		1	&3	&2	&6	&4	&9	&8	&7	&5	\\ \hhline{|=|=|=#=|=|=#=|=|=|}
		4	&7	&3	&9	&1	&2	&5	&6	&8	\\ \hline
		9	&6	&1	&7	&8	&5	&4	&3	&2	\\ \hline
		2	&5	&8	&3	&6	&4	&7	&9	&1	\\ \hhline{|=|=|=#=|=|=#=|=|=|}
		7	&2	&4	&1	&9	&8	&3	&5	&6	\\ \hline
		6	&1	&5	&2	&7	&3	&9	&8	&4	\\ \hline
		3	&8	&9	&4	&5	&6	&1	&2	&7	\\ \hline
	\end{tabular}
\end{table}

\end{document}
